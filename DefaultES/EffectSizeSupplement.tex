\documentclass[]{tufte-handout}

% ams
\usepackage{amssymb,amsmath}

\usepackage{ifxetex,ifluatex}
\usepackage{fixltx2e} % provides \textsubscript
\ifnum 0\ifxetex 1\fi\ifluatex 1\fi=0 % if pdftex
  \usepackage[T1]{fontenc}
  \usepackage[utf8]{inputenc}
\else % if luatex or xelatex
  \makeatletter
  \@ifpackageloaded{fontspec}{}{\usepackage{fontspec}}
  \makeatother
  \defaultfontfeatures{Ligatures=TeX,Scale=MatchLowercase}
  \makeatletter
  \@ifpackageloaded{soul}{
     \renewcommand\allcapsspacing[1]{{\addfontfeature{LetterSpace=15}#1}}
     \renewcommand\smallcapsspacing[1]{{\addfontfeature{LetterSpace=10}#1}}
   }{}
  \makeatother

\fi

% graphix
\usepackage{graphicx}
\setkeys{Gin}{width=\linewidth,totalheight=\textheight,keepaspectratio}

% booktabs
\usepackage{booktabs}

% url
\usepackage{url}

% hyperref
\usepackage{hyperref}

% units.
\usepackage{units}


\setcounter{secnumdepth}{-1}

% citations
\usepackage{natbib}
\bibliographystyle{plainnat}

% pandoc syntax highlighting
\usepackage{color}
\usepackage{fancyvrb}
\newcommand{\VerbBar}{|}
\newcommand{\VERB}{\Verb[commandchars=\\\{\}]}
\DefineVerbatimEnvironment{Highlighting}{Verbatim}{commandchars=\\\{\}}
% Add ',fontsize=\small' for more characters per line
\newenvironment{Shaded}{}{}
\newcommand{\AlertTok}[1]{\textcolor[rgb]{1.00,0.00,0.00}{\textbf{#1}}}
\newcommand{\AnnotationTok}[1]{\textcolor[rgb]{0.38,0.63,0.69}{\textbf{\textit{#1}}}}
\newcommand{\AttributeTok}[1]{\textcolor[rgb]{0.49,0.56,0.16}{#1}}
\newcommand{\BaseNTok}[1]{\textcolor[rgb]{0.25,0.63,0.44}{#1}}
\newcommand{\BuiltInTok}[1]{#1}
\newcommand{\CharTok}[1]{\textcolor[rgb]{0.25,0.44,0.63}{#1}}
\newcommand{\CommentTok}[1]{\textcolor[rgb]{0.38,0.63,0.69}{\textit{#1}}}
\newcommand{\CommentVarTok}[1]{\textcolor[rgb]{0.38,0.63,0.69}{\textbf{\textit{#1}}}}
\newcommand{\ConstantTok}[1]{\textcolor[rgb]{0.53,0.00,0.00}{#1}}
\newcommand{\ControlFlowTok}[1]{\textcolor[rgb]{0.00,0.44,0.13}{\textbf{#1}}}
\newcommand{\DataTypeTok}[1]{\textcolor[rgb]{0.56,0.13,0.00}{#1}}
\newcommand{\DecValTok}[1]{\textcolor[rgb]{0.25,0.63,0.44}{#1}}
\newcommand{\DocumentationTok}[1]{\textcolor[rgb]{0.73,0.13,0.13}{\textit{#1}}}
\newcommand{\ErrorTok}[1]{\textcolor[rgb]{1.00,0.00,0.00}{\textbf{#1}}}
\newcommand{\ExtensionTok}[1]{#1}
\newcommand{\FloatTok}[1]{\textcolor[rgb]{0.25,0.63,0.44}{#1}}
\newcommand{\FunctionTok}[1]{\textcolor[rgb]{0.02,0.16,0.49}{#1}}
\newcommand{\ImportTok}[1]{#1}
\newcommand{\InformationTok}[1]{\textcolor[rgb]{0.38,0.63,0.69}{\textbf{\textit{#1}}}}
\newcommand{\KeywordTok}[1]{\textcolor[rgb]{0.00,0.44,0.13}{\textbf{#1}}}
\newcommand{\NormalTok}[1]{#1}
\newcommand{\OperatorTok}[1]{\textcolor[rgb]{0.40,0.40,0.40}{#1}}
\newcommand{\OtherTok}[1]{\textcolor[rgb]{0.00,0.44,0.13}{#1}}
\newcommand{\PreprocessorTok}[1]{\textcolor[rgb]{0.74,0.48,0.00}{#1}}
\newcommand{\RegionMarkerTok}[1]{#1}
\newcommand{\SpecialCharTok}[1]{\textcolor[rgb]{0.25,0.44,0.63}{#1}}
\newcommand{\SpecialStringTok}[1]{\textcolor[rgb]{0.73,0.40,0.53}{#1}}
\newcommand{\StringTok}[1]{\textcolor[rgb]{0.25,0.44,0.63}{#1}}
\newcommand{\VariableTok}[1]{\textcolor[rgb]{0.10,0.09,0.49}{#1}}
\newcommand{\VerbatimStringTok}[1]{\textcolor[rgb]{0.25,0.44,0.63}{#1}}
\newcommand{\WarningTok}[1]{\textcolor[rgb]{0.38,0.63,0.69}{\textbf{\textit{#1}}}}

% longtable
\usepackage{longtable,booktabs}

% multiplecol
\usepackage{multicol}

% strikeout
\usepackage[normalem]{ulem}

% morefloats
\usepackage{morefloats}


% tightlist macro required by pandoc >= 1.14
\providecommand{\tightlist}{%
  \setlength{\itemsep}{0pt}\setlength{\parskip}{0pt}}

% title / author / date
\title{How to Calculate and Report Effect Sizes}
\author{Aaron R. Caldwell}
\date{2020-10-26}


\begin{document}

\maketitle




\hypertarget{introduction}{%
\section{Introduction}\label{introduction}}

In our manuscript, we argued that sport and exercise scientists should
avoid the temptation of reporting a ``default'' effect size. Instead, we
suggested that the effect size be chosen on a case-by-case basis and
reported in a way that facilitates the appropriate communication of
study results. With that said, we realize many sport scientists may not
know how to calculate these effect sizes in a efficient manner.
Therefore, we have written this document to show how \textbf{R}
\citep{R-base} can be used for these calculations.

\hypertarget{loading-the-right-tools}{%
\subsection{Loading the Right Tools}\label{loading-the-right-tools}}

None of the following packages are `necessary', and all of these
analyses can be performed in `base' \texttt{R}. But, I use these
packages to make the analysis pipeline simpler and a little `cleaner'
too. In total, I need the \texttt{tidyverse} \citep{tidyverse},
\texttt{boot} \citep{boot}, \texttt{broom} \citep{broom}, and
\texttt{MASS} \citep{MASS} packages. You will likely not need
\texttt{MASS} or \texttt{broom} for your own analyses, but it is
necessary for this document in order to create a simulated dataset
(\texttt{MASS}) and make the tables in the document (\texttt{broom}).

\begin{Shaded}
\begin{Highlighting}[]
\CommentTok{#LOAD PACKAGES}
\KeywordTok{library}\NormalTok{(tidyverse) }
\KeywordTok{library}\NormalTok{(boot) }
\KeywordTok{library}\NormalTok{(MASS) }
\KeywordTok{library}\NormalTok{(broom)}
\end{Highlighting}
\end{Shaded}

\hypertarget{example-1-dotvo_2-data}{%
\section{\texorpdfstring{Example 1: \(\dot{V}O_2\)
Data}{Example 1: \textbackslash dot\{V\}O\_2 Data}}\label{example-1-dotvo_2-data}}

\hypertarget{generate-some-data}{%
\subsection{Generate Some Data}\label{generate-some-data}}

Okay, now I can generate data \emph{similar} to what we used in the
manuscipt. First, let us create an ``Interpretatable Raw Differences''
example. In this example, we have a pre-to-post design of athletes where
we are measuring \(\dot{V}O_2\) (L\(\cdot\)min\textsuperscript{-1}).
Further, we assume we are sampling from a population with a standard
deviation of 0.2, pre-post correlation of 0.9, and a mean increase of
0.3 (L\(\cdot\)min\textsuperscript{-1}).

\begin{Shaded}
\begin{Highlighting}[]
\KeywordTok{set.seed}\NormalTok{(}\DecValTok{20200303}\NormalTok{)}
\NormalTok{var_raw =}\StringTok{ }\FloatTok{.2}\OperatorTok{^}\DecValTok{2}
\NormalTok{cor_raw =}\StringTok{ }\KeywordTok{matrix}\NormalTok{(}\KeywordTok{c}\NormalTok{(}\DecValTok{1}\NormalTok{,.}\DecValTok{8}\NormalTok{,.}\DecValTok{8}\NormalTok{,}\DecValTok{1}\NormalTok{),}\DataTypeTok{nrow=}\DecValTok{2}\NormalTok{)}
\NormalTok{Sigma_raw =}\StringTok{ }\NormalTok{cor_raw}\OperatorTok{*}\NormalTok{var_raw}
\NormalTok{df_raw =}\StringTok{ }\KeywordTok{mvrnorm}\NormalTok{(}
  \DataTypeTok{n =} \DecValTok{10}\NormalTok{,}
  \DataTypeTok{Sigma =}\NormalTok{ Sigma_raw,}
  \DataTypeTok{mu =} \KeywordTok{c}\NormalTok{(}\FloatTok{3.9}\NormalTok{, }\FloatTok{4.1}\NormalTok{)}
\NormalTok{  ) }\OperatorTok\StringTok{ }
\StringTok{  }\KeywordTok{as.data.frame}\NormalTok{() }\OperatorTok
\StringTok{  }\KeywordTok{rename}\NormalTok{(}\DataTypeTok{pre =}\NormalTok{ V1,}
         \DataTypeTok{post =}\NormalTok{ V2)}

\CommentTok{#Add a change score column}
\NormalTok{df_raw =}\StringTok{ }\NormalTok{df_raw }\OperatorTok\StringTok{ }\KeywordTok{mutate}\NormalTok{(}\DataTypeTok{pre =} \KeywordTok{round}\NormalTok{(pre,}\DecValTok{2}\NormalTok{),}
                           \DataTypeTok{post =} \KeywordTok{round}\NormalTok{(post,}\DecValTok{2}\NormalTok{)) }\OperatorTok
\StringTok{  }\KeywordTok{mutate}\NormalTok{(}\DataTypeTok{diff =}\NormalTok{ post}\OperatorTok{-}\NormalTok{pre)}
\end{Highlighting}
\end{Shaded}

\begin{longtable}[]{@{}rrr@{}}
\caption{Raw Differences Data}\tabularnewline
\toprule
pre & post & diff\tabularnewline
\midrule
\endfirsthead
\toprule
pre & post & diff\tabularnewline
\midrule
\endhead
3.85 & 4.15 & 0.30\tabularnewline
3.75 & 4.11 & 0.36\tabularnewline
4.03 & 4.14 & 0.11\tabularnewline
3.95 & 4.08 & 0.13\tabularnewline
4.22 & 4.40 & 0.18\tabularnewline
4.20 & 4.11 & -0.09\tabularnewline
3.74 & 3.99 & 0.25\tabularnewline
4.13 & 4.17 & 0.04\tabularnewline
3.98 & 4.05 & 0.07\tabularnewline
4.05 & 4.20 & 0.15\tabularnewline
\bottomrule
\end{longtable}

\hypertarget{the-simple-analysis}{%
\subsection{The Simple Analysis}\label{the-simple-analysis}}

Now that we have our data, we can perform a simple \emph{t}-test on the
difference scores.

\begin{Shaded}
\begin{Highlighting}[]
\NormalTok{t_raw =}\StringTok{ }\KeywordTok{t.test}\NormalTok{(df_raw}\OperatorTok{$}\NormalTok{diff)}
\NormalTok{knitr}\OperatorTok{::}\KeywordTok{kable}\NormalTok{(}\KeywordTok{tidy}\NormalTok{(t_raw),}\DataTypeTok{caption=}\StringTok{"Raw Differences t-test"}\NormalTok{,}
             \DataTypeTok{digits =} \DecValTok{4}\NormalTok{)}
\end{Highlighting}
\end{Shaded}

\begin{longtable}[]{@{}rrrrrrll@{}}
\caption{Raw Differences t-test}\tabularnewline
\toprule
estimate & statistic & p.value & parameter & conf.low & conf.high &
method & alternative\tabularnewline
\midrule
\endfirsthead
\toprule
estimate & statistic & p.value & parameter & conf.low & conf.high &
method & alternative\tabularnewline
\midrule
\endhead
0.15 & 3.6075 & 0.0057 & 9 & 0.0559 & 0.2441 & One Sample t-test &
two.sided\tabularnewline
\bottomrule
\end{longtable}

Okay, so we now have our \emph{t}-statistics, \emph{p}-values, mean
differences, and confidence intervals. Pretty painless to get this far,
but maybe we want (or need) to get the standardized mean difference
(SMD) and common langauge effect sizes (CLES).

\hypertarget{lets-bootstrap}{%
\subsection{Let's Bootstrap}\label{lets-bootstrap}}

While it is easy to calculate the SMD or CLES in R, it is quite another
thing to calculate their confidence intervals (CI). Unlike the mean
difference, the formulae for the CI of the SMD are convoluted (see
\citet{hedges1981}). Thus, here, we rely on the bootstrap to generate
CIs.\footnote{There are some formula for calculating the standard error,
  and therefore confidence intervals, of SMDs. Many of these
  calculations are contained in a review by
  \citet{Nakagawa_Cuthill_2007}. However, the required formula is
  dependent on the SMD being chosen. For example, the SE calculation
  differs from Cohen's d and Hedges' g, despite the later being a
  corrected version of the former. To avoid potential miscalculations,
  we strongly advocate for the use of bootstrapping methods.}

First, we must create a function to calculate all the SMD statistics we
would like to obtain.

\begin{Shaded}
\begin{Highlighting}[]
\CommentTok{# function to obtain R-Squared from the data}
\NormalTok{SMD <-}\StringTok{ }\ControlFlowTok{function}\NormalTok{(data, indices) \{}
\NormalTok{  d <-}\StringTok{ }\NormalTok{data[indices,] }\CommentTok{# allows boot to select sample}
\NormalTok{  sd_pre =}\StringTok{ }\KeywordTok{sd}\NormalTok{(d}\OperatorTok{$}\NormalTok{pre, }\DataTypeTok{na.rm =} \OtherTok{TRUE}\NormalTok{)}
\NormalTok{  sd_post =}\StringTok{ }\KeywordTok{sd}\NormalTok{(d}\OperatorTok{$}\NormalTok{post, }\DataTypeTok{na.rm =} \OtherTok{TRUE}\NormalTok{)}
\NormalTok{  sd_av =}\StringTok{ }\NormalTok{(sd_pre}\OperatorTok{+}\NormalTok{sd_post)}\OperatorTok{/}\DecValTok{2}
\NormalTok{  m_diff =}\StringTok{ }\KeywordTok{mean}\NormalTok{(d}\OperatorTok{$}\NormalTok{diff, }\DataTypeTok{na.rm =} \OtherTok{TRUE}\NormalTok{)}
\NormalTok{  sd_diff =}\StringTok{ }\KeywordTok{sd}\NormalTok{(d}\OperatorTok{$}\NormalTok{diff, }\DataTypeTok{na.rm =} \OtherTok{TRUE}\NormalTok{)}
\NormalTok{  cor_prepost =}\StringTok{ }\KeywordTok{cor}\NormalTok{(d}\OperatorTok{$}\NormalTok{pre,d}\OperatorTok{$}\NormalTok{post,}
                    \DataTypeTok{method =} \StringTok{"pearson"}\NormalTok{)}
\NormalTok{  dz =}\StringTok{ }\NormalTok{m_diff}\OperatorTok{/}\NormalTok{sd_diff}
\NormalTok{  dav =}\StringTok{ }\NormalTok{m_diff}\OperatorTok{/}\NormalTok{sd_av}
\NormalTok{  glass =}\StringTok{ }\NormalTok{m_diff}\OperatorTok{/}\NormalTok{sd_pre}
\NormalTok{  drm =}\StringTok{ }\NormalTok{m_diff}\OperatorTok{/}\KeywordTok{sqrt}\NormalTok{((sd_pre}\OperatorTok{^}\DecValTok{2}\OperatorTok{+}\NormalTok{sd_post}\OperatorTok{^}\DecValTok{2}\NormalTok{)}\OperatorTok{-}\NormalTok{(}\DecValTok{2}\OperatorTok{*}\NormalTok{cor_prepost}\OperatorTok{*}\NormalTok{sd_pre}\OperatorTok{*}\NormalTok{sd_post))}\OperatorTok{*}\KeywordTok{sqrt}\NormalTok{(}\DecValTok{2}\OperatorTok{*}\NormalTok{(}\DecValTok{1}\OperatorTok{-}\NormalTok{cor_prepost))}
\NormalTok{  CLES =}\StringTok{ }\KeywordTok{pnorm}\NormalTok{(dz)}
\NormalTok{  result =}\StringTok{ }\KeywordTok{c}\NormalTok{(dz,drm,dav,glass,CLES)}
  
\NormalTok{\} }
\end{Highlighting}
\end{Shaded}

Second, we can start bootstrapping these statistics. To do so, we select
each value based on their index (1-5) from the function above. We must
also specify the \emph{type} of bootstrap CI. In this case, I have
decided to report the bias-corrected and accelerated (BCa) intervals as
they will likely provide our most accurate estimate.

\begin{Shaded}
\begin{Highlighting}[]
\NormalTok{raw_boot =}\StringTok{ }\KeywordTok{boot}\NormalTok{(df_raw, SMD, }\DataTypeTok{R =} \DecValTok{2000}\NormalTok{)}
\CommentTok{#Extract the values}
\NormalTok{dz_raw =}\StringTok{ }\KeywordTok{boot.ci}\NormalTok{(raw_boot, }\DataTypeTok{type=}\StringTok{"bca"}\NormalTok{, }\DataTypeTok{index=}\DecValTok{1}\NormalTok{)}
\NormalTok{drm_raw =}\StringTok{ }\KeywordTok{boot.ci}\NormalTok{(raw_boot, }\DataTypeTok{type=}\StringTok{"bca"}\NormalTok{, }\DataTypeTok{index=}\DecValTok{2}\NormalTok{)}
\NormalTok{dav_raw =}\StringTok{ }\KeywordTok{boot.ci}\NormalTok{(raw_boot, }\DataTypeTok{type=}\StringTok{"bca"}\NormalTok{, }\DataTypeTok{index=}\DecValTok{3}\NormalTok{)}
\NormalTok{glass_raw =}\StringTok{ }\KeywordTok{boot.ci}\NormalTok{(raw_boot, }\DataTypeTok{type=}\StringTok{"bca"}\NormalTok{, }\DataTypeTok{index=}\DecValTok{4}\NormalTok{)}
\NormalTok{CLES_raw =}\StringTok{ }\KeywordTok{boot.ci}\NormalTok{(raw_boot, }\DataTypeTok{type=}\StringTok{"bca"}\NormalTok{, }\DataTypeTok{index=}\DecValTok{5}\NormalTok{)}
\end{Highlighting}
\end{Shaded}

From these statistics, we can extract the effect size estimate by
calling on the \texttt{\$t0} part of the saved object.

For example:

\begin{Shaded}
\begin{Highlighting}[]
\NormalTok{dz_raw}\OperatorTok{$}\NormalTok{t0}
\end{Highlighting}
\end{Shaded}

\begin{verbatim}
## [1] 1.140795
\end{verbatim}

And the upper limit and lower limit of the 95\% CI can also be found in
the \texttt{dz\_raw\$bca} object. The last and second-to-last numbers
represent the upper limit and lower limit of the 95\% CI.

\begin{Shaded}
\begin{Highlighting}[]
\NormalTok{dz_raw}\OperatorTok{$}\NormalTok{bca}
\end{Highlighting}
\end{Shaded}

\begin{verbatim}
##      conf                                
## [1,] 0.95 3.44 1806.43 0.1956553 1.991893
\end{verbatim}

Finally, we can put all of the SMD calculations in a summary table.

\begin{longtable}[]{@{}lrrr@{}}
\caption{Effect Sizes for `Raw' Differences Dataset}\tabularnewline
\toprule
Effect Size & Estimate & Lower Limit & Upper Limit\tabularnewline
\midrule
\endfirsthead
\toprule
Effect Size & Estimate & Lower Limit & Upper Limit\tabularnewline
\midrule
\endhead
d(z) & 1.1408 & 0.1957 & 1.9919\tabularnewline
d(rm) & 0.9688 & 0.3886 & 1.5554\tabularnewline
d(av) & 1.0689 & 0.4172 & 1.8032\tabularnewline
Glass's Delta & 0.8771 & 0.2360 & 1.2969\tabularnewline
CLES & 0.8730 & 0.5714 & 0.9753\tabularnewline
\bottomrule
\end{longtable}

\hypertarget{example-2-vas-data}{%
\section{Example 2: VAS Data}\label{example-2-vas-data}}

\hypertarget{generate-some-data-1}{%
\subsection{Generate Some Data}\label{generate-some-data-1}}

Now, we want to create an ``Uninterpretable Raw Differences'' dataset.
This time we are creating a dummy dataset of VAS scores with 15 `point'
reduction and correlation of 0.7 on the latent scale. Since sensation
data are typically distributed log-normal, we will also work with a
latent normal distribution.

\begin{Shaded}
\begin{Highlighting}[]
\KeywordTok{set.seed}\NormalTok{(}\DecValTok{20201113}\NormalTok{)}

\NormalTok{v =}\StringTok{ }\DecValTok{13}\OperatorTok{^}\DecValTok{2} \CommentTok{# observed variance (log-normal)}
\NormalTok{m =}\StringTok{ }\KeywordTok{c}\NormalTok{(}\DecValTok{50}\NormalTok{,}\DecValTok{35}\NormalTok{) }\CommentTok{# observed means (log-normal)}
\NormalTok{phi =}\StringTok{ }\KeywordTok{sqrt}\NormalTok{(v }\OperatorTok{+}\StringTok{ }\NormalTok{m}\OperatorTok{^}\DecValTok{2}\NormalTok{)}
\NormalTok{sd_stan =}\StringTok{ }\KeywordTok{sqrt}\NormalTok{(}\KeywordTok{log}\NormalTok{(phi}\OperatorTok{^}\DecValTok{2}\OperatorTok{/}\NormalTok{m}\OperatorTok{^}\DecValTok{2}\NormalTok{)) }\CommentTok{# latent SD (normal)}
\NormalTok{mu_stan =}\StringTok{ }\KeywordTok{log}\NormalTok{(m}\OperatorTok{^}\DecValTok{2}\OperatorTok{/}\NormalTok{phi) }\CommentTok{# latent mean (normal)}
\NormalTok{cor_stan =}\StringTok{ }\KeywordTok{matrix}\NormalTok{(}\KeywordTok{c}\NormalTok{(}\DecValTok{1}\NormalTok{,.}\DecValTok{7}\NormalTok{,.}\DecValTok{7}\NormalTok{,}\DecValTok{1}\NormalTok{),}\DataTypeTok{nrow =} \DecValTok{2}\NormalTok{)}
\NormalTok{Sigma_stan =}\StringTok{ }\NormalTok{(sd_stan }\OperatorTok\StringTok{ }\KeywordTok{t}\NormalTok{(sd_stan)) }\OperatorTok{*}\StringTok{ }\NormalTok{cor_stan}

\CommentTok{# Generate latent ratings}
\NormalTok{df_stan =}\StringTok{ }\KeywordTok{mvrnorm}\NormalTok{(}
  \DataTypeTok{n =} \DecValTok{10}\NormalTok{,}
  \DataTypeTok{Sigma =}\NormalTok{ Sigma_stan,}
  \DataTypeTok{mu =}\NormalTok{ mu_stan}
\NormalTok{) }\OperatorTok\StringTok{ }
\StringTok{  }\KeywordTok{as.data.frame}\NormalTok{() }\OperatorTok
\StringTok{  }\KeywordTok{rename}\NormalTok{(}\DataTypeTok{pre =}\NormalTok{ V1,}
         \DataTypeTok{post =}\NormalTok{ V2)}

\CommentTok{#Add a change score column}
\CommentTok{# Latent perceptions}
\NormalTok{df_stan =}\StringTok{ }\NormalTok{df_stan }\OperatorTok\StringTok{ }\KeywordTok{mutate}\NormalTok{(}\DataTypeTok{pre =} \KeywordTok{round}\NormalTok{(pre,}\DecValTok{2}\NormalTok{),}
                           \DataTypeTok{post =} \KeywordTok{round}\NormalTok{(post,}\DecValTok{2}\NormalTok{)) }\OperatorTok
\StringTok{  }\KeywordTok{mutate}\NormalTok{(}\DataTypeTok{diff =}\NormalTok{ post }\OperatorTok{-}\StringTok{ }\NormalTok{pre)}

\CommentTok{# Observed perceptions (log-normal)}
\NormalTok{df_stan_obs =}\StringTok{ }\KeywordTok{exp}\NormalTok{(df_stan[,}\DecValTok{1}\OperatorTok{:}\DecValTok{2}\NormalTok{]) }\OperatorTok\StringTok{ }
\StringTok{  }\KeywordTok{mutate}\NormalTok{(}\DataTypeTok{diff =}\NormalTok{ post }\OperatorTok{-}\StringTok{ }\NormalTok{pre)}
\end{Highlighting}
\end{Shaded}

\begin{longtable}[]{@{}rrr@{}}
\caption{Log-transformed Difference Data}\tabularnewline
\toprule
pre & post & diff\tabularnewline
\midrule
\endfirsthead
\toprule
pre & post & diff\tabularnewline
\midrule
\endhead
3.52 & 2.93 & -0.59\tabularnewline
3.83 & 3.34 & -0.49\tabularnewline
3.50 & 3.53 & 0.03\tabularnewline
4.14 & 3.60 & -0.54\tabularnewline
3.56 & 2.97 & -0.59\tabularnewline
3.99 & 3.23 & -0.76\tabularnewline
3.71 & 2.83 & -0.88\tabularnewline
4.01 & 3.49 & -0.52\tabularnewline
3.74 & 3.38 & -0.36\tabularnewline
4.09 & 3.43 & -0.66\tabularnewline
\bottomrule
\end{longtable}

\hypertarget{the-simple-analysis-1}{%
\subsection{The Simple Analysis}\label{the-simple-analysis-1}}

Now that we have our data, we can perform a simple \emph{t}-test on the
difference scores, both latent and observed.

\begin{Shaded}
\begin{Highlighting}[]
\NormalTok{t_stan_latent =}\StringTok{ }\KeywordTok{t.test}\NormalTok{(df_stan}\OperatorTok{$}\NormalTok{diff)}
\NormalTok{knitr}\OperatorTok{::}\KeywordTok{kable}\NormalTok{(}\KeywordTok{tidy}\NormalTok{(t_stan_latent),}\DataTypeTok{caption=}\StringTok{"Log-transformed Differences t-test"}\NormalTok{,}
             \DataTypeTok{digits =} \DecValTok{4}\NormalTok{)}
\end{Highlighting}
\end{Shaded}

\begin{longtable}[]{@{}rrrrrrll@{}}
\caption{Log-transformed Differences t-test}\tabularnewline
\toprule
estimate & statistic & p.value & parameter & conf.low & conf.high &
method & alternative\tabularnewline
\midrule
\endfirsthead
\toprule
estimate & statistic & p.value & parameter & conf.low & conf.high &
method & alternative\tabularnewline
\midrule
\endhead
-0.536 & -6.8978 & 1e-04 & 9 & -0.7118 & -0.3602 & One Sample t-test &
two.sided\tabularnewline
\bottomrule
\end{longtable}

\begin{Shaded}
\begin{Highlighting}[]
\NormalTok{t_stan_obs =}\StringTok{ }\KeywordTok{t.test}\NormalTok{(df_stan_obs}\OperatorTok{$}\NormalTok{diff)}
\NormalTok{knitr}\OperatorTok{::}\KeywordTok{kable}\NormalTok{(}\KeywordTok{tidy}\NormalTok{(t_stan_obs),}\DataTypeTok{caption=}\StringTok{"Observed Differences t-test"}\NormalTok{,}
             \DataTypeTok{digits =} \DecValTok{4}\NormalTok{)}
\end{Highlighting}
\end{Shaded}

\begin{longtable}[]{@{}rrrrrrll@{}}
\caption{Observed Differences t-test}\tabularnewline
\toprule
estimate & statistic & p.value & parameter & conf.low & conf.high &
method & alternative\tabularnewline
\midrule
\endfirsthead
\toprule
estimate & statistic & p.value & parameter & conf.low & conf.high &
method & alternative\tabularnewline
\midrule
\endhead
-19.0403 & -6.6056 & 1e-04 & 9 & -25.5608 & -12.5197 & One Sample t-test
& two.sided\tabularnewline
\bottomrule
\end{longtable}

Since the transformation is nonlinear, the expected difference in the
observed data is not a simple transformation of the expected difference
of the log-transformed data.

\hypertarget{lets-bootstrap-1}{%
\subsection{Let's Bootstrap}\label{lets-bootstrap-1}}

Using the same methods and functions as before, we will use the
bootstrap to calculate CIs for each of the SMDs and CLES.

\begin{Shaded}
\begin{Highlighting}[]
\CommentTok{#Repeat for the log-transformed dataset}
\NormalTok{stan_boot =}\StringTok{ }\KeywordTok{boot}\NormalTok{(df_stan, SMD, }\DataTypeTok{R =} \DecValTok{2000}\NormalTok{)}
\NormalTok{dz_stan =}\StringTok{ }\KeywordTok{boot.ci}\NormalTok{(stan_boot, }\DataTypeTok{type=}\StringTok{"bca"}\NormalTok{, }\DataTypeTok{index=}\DecValTok{1}\NormalTok{)}
\NormalTok{drm_stan =}\StringTok{ }\KeywordTok{boot.ci}\NormalTok{(stan_boot, }\DataTypeTok{type=}\StringTok{"bca"}\NormalTok{, }\DataTypeTok{index=}\DecValTok{2}\NormalTok{)}
\NormalTok{dav_stan =}\StringTok{ }\KeywordTok{boot.ci}\NormalTok{(stan_boot, }\DataTypeTok{type=}\StringTok{"bca"}\NormalTok{, }\DataTypeTok{index=}\DecValTok{3}\NormalTok{)}
\NormalTok{glass_stan =}\StringTok{ }\KeywordTok{boot.ci}\NormalTok{(stan_boot, }\DataTypeTok{type=}\StringTok{"bca"}\NormalTok{, }\DataTypeTok{index=}\DecValTok{4}\NormalTok{)}
\NormalTok{CLES_stan =}\StringTok{ }\KeywordTok{boot.ci}\NormalTok{(stan_boot, }\DataTypeTok{type=}\StringTok{"bca"}\NormalTok{, }\DataTypeTok{index=}\DecValTok{5}\NormalTok{)}
\end{Highlighting}
\end{Shaded}

\begin{verbatim}
## Warning in norm.inter(t, adj.alpha): extreme order statistics used as endpoints
\end{verbatim}

\begin{Shaded}
\begin{Highlighting}[]
\CommentTok{#Repeat for the observed dataset}
\NormalTok{stan_obs_boot =}\StringTok{ }\KeywordTok{boot}\NormalTok{(df_stan_obs, SMD, }\DataTypeTok{R =} \DecValTok{2000}\NormalTok{)}
\NormalTok{dz_stan_obs =}\StringTok{ }\KeywordTok{boot.ci}\NormalTok{(stan_obs_boot, }\DataTypeTok{type=}\StringTok{"bca"}\NormalTok{, }\DataTypeTok{index=}\DecValTok{1}\NormalTok{)}
\NormalTok{drm_stan_obs =}\StringTok{ }\KeywordTok{boot.ci}\NormalTok{(stan_obs_boot, }\DataTypeTok{type=}\StringTok{"bca"}\NormalTok{, }\DataTypeTok{index=}\DecValTok{2}\NormalTok{)}
\NormalTok{dav_stan_obs =}\StringTok{ }\KeywordTok{boot.ci}\NormalTok{(stan_obs_boot, }\DataTypeTok{type=}\StringTok{"bca"}\NormalTok{, }\DataTypeTok{index=}\DecValTok{3}\NormalTok{)}
\NormalTok{glass_stan_obs =}\StringTok{ }\KeywordTok{boot.ci}\NormalTok{(stan_obs_boot, }\DataTypeTok{type=}\StringTok{"bca"}\NormalTok{, }\DataTypeTok{index=}\DecValTok{4}\NormalTok{)}
\NormalTok{CLES_stan_obs =}\StringTok{ }\KeywordTok{boot.ci}\NormalTok{(stan_obs_boot, }\DataTypeTok{type=}\StringTok{"bca"}\NormalTok{, }\DataTypeTok{index=}\DecValTok{5}\NormalTok{)}
\end{Highlighting}
\end{Shaded}

From these statistics, we can extract the effect size estimate by
calling on the \texttt{\$t0} part of the saved object.

For example, if we look at the latent SMD:

\begin{Shaded}
\begin{Highlighting}[]
\NormalTok{dz_stan}\OperatorTok{$}\NormalTok{t0}
\end{Highlighting}
\end{Shaded}

\begin{verbatim}
## [1] -2.181274
\end{verbatim}

And the upper limit and lower limit of the 95\% CI can also be found in
the \texttt{dz\_stan\$bca} object. The last and second-to-last numbers
represent the upper limit and lower limit of the 95\% CI.

\begin{Shaded}
\begin{Highlighting}[]
\NormalTok{dz_stan}\OperatorTok{$}\NormalTok{bca}
\end{Highlighting}
\end{Shaded}

\begin{verbatim}
##      conf                                    
## [1,] 0.95 253.93 1999.86 -4.571156 -0.8416896
\end{verbatim}

Finally, we can put all of the SMD calculations in a summary table.
Note, the effect sizes in the 2nd table are negative; that is because
there was a decrease from pre-to-post, and the CLES should be reported
as \(1-CLES\) from the table.

\begin{longtable}[]{@{}lrrr@{}}
\caption{Effect Sizes for Log-transformed Differences}\tabularnewline
\toprule
Effect Size & Estimate & Lower Limit & Upper Limit\tabularnewline
\midrule
\endfirsthead
\toprule
Effect Size & Estimate & Lower Limit & Upper Limit\tabularnewline
\midrule
\endhead
d(z) & -2.1813 & -4.5712 & -0.8417\tabularnewline
d(rm) & -2.0781 & -2.7957 & -1.3713\tabularnewline
d(av) & -2.0925 & -2.7204 & -1.2903\tabularnewline
Glass's Delta & -2.2349 & -3.1933 & -1.2703\tabularnewline
CLES & 0.0146 & 0.0000 & 0.2070\tabularnewline
\bottomrule
\end{longtable}

\begin{longtable}[]{@{}lrrr@{}}
\caption{Effect Sizes for Observed Differences}\tabularnewline
\toprule
Effect Size & Estimate & Lower Limit & Upper Limit\tabularnewline
\midrule
\endfirsthead
\toprule
Effect Size & Estimate & Lower Limit & Upper Limit\tabularnewline
\midrule
\endhead
d(z) & -2.0889 & -3.8253 & -0.8166\tabularnewline
d(rm) & -1.9477 & -3.0301 & -1.2459\tabularnewline
d(av) & -2.1265 & -2.9616 & -1.3368\tabularnewline
Glass's Delta & -1.7276 & -2.3951 & -1.0051\tabularnewline
CLES & 0.0184 & 0.0001 & 0.2142\tabularnewline
\bottomrule
\end{longtable}

\newpage

\renewcommand\refname{References}
\bibliography{refs.bib}



\end{document}
